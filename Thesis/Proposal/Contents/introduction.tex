\section{Introduction}
The use of drones as a means of automating tasks that are currently being 
performed manually are becoming ubiquitous 
\cite{application1cardiac, application2forestfires, blindrunnersdrone,PowerlineFollower}. 
These applications can vary from using drones as a means of package delivery, 
object tracking and entertainment purposes. Many of these applications require 
behavior from the device that includes autonomy, which in itself is a very complex
defined set of behaviors. Many applications for the drone
still require manually controlling of the device in order to perform intelligent behavior. 
The rise of new implementations of autonomous drones therefore provides the challenge 
to find new practices in which automating these tasks is possible. 

Many of these applications are based on the behavior of tracking an object. 
The task of keeping a certain object in view of the drone, is a type of behavior 
that can be used in a variety of practices. The most effective solution would 
be to have the drone keep an object in its field of vision using a camera.
The tracking of an object has recently been applied to follow a powerline 
in order to perform maintenance \cite{PowerlineFollower}. 
This implementation raises the question whether this would also be possible 
for the tracking of persons in a variety of environments. Already established practices
have been the tracking of athletes while filming \cite{application3sports} and 
guiding blind people in the streets \cite{blindrunnersdrone}. In these environments, 
the drones are able to accurately track and maintain their distance from an object, 
while simultaneously keeping the object in sight. 

However, some applications require smaller devices and therefore, lower computational models
in order to achieve the same type of behavior. As an example, the purpose
f a drone camera that can automatically take pictures and/or record an 
individual in indoor environments. Smaller devices have computational limitations 
as well as physical limitations that need to be taken into account. 
These devices, referred to as Embedded Devices (ED), can be defined as having been 
produced with a specific function in mind \cite{embeddedsystem}. Per definition, 
these devices can be categorized as everything between a coffeemaker and mobile 
phones. In most cases, these EDs lack GPU power, which limits them from performing 
heavy computations. Therefore, when developing models for these devices, it is 
necessary to sacrifices model accuracy for overall speed of inference. At the same time, certain 
technologies used in obstacle detection become problematic since these 
could outweigh the carrying capacity of the drone. Therefore, many of the techniques 
used in developing the follow-me drones, can not be used in indoor applications. 
The need for the application of these technologies on EDs is therefore still relevant. 

Recently, the rise in using Deep Neural Networks applied to robotics could show 
promising possibilities for these problems. These neural 
networks are used in the context of Machine Learning \cite{neuralnets}.
A specific way in which these can be applied, is in a Reinforcement Learning (RL) 
application. This method is most useful in applications where there is a necessity 
to map actions to certain environmental states and the annotation of these state is 
too large to perform manually. These RL agents thrive in decision 
making problems where the agent is able to explore the environment itself and correct 
itself according to a pre-determined reward function. Applying this to the use of 
indoor follow-me drones would be an interesting challenge. Furthermore, a pitfall is that 
training in real-time is an unreasonable endeavour since RL requires a vast amount of 
training instances for the learning process. Therefore, simulation environments
will be considered as a possibility for the algorithm to learn in an offline environment 
before being deployed in the real-world. Optimizing the process of training and deployment of
an agent to perform follow-me behavior on a drone is therefore an active field of 
research. 